In this lecture we covered Hamiltonian mechanics and vector fields, as well as the Poisson bracket.
The goal of this lecture was to introduce some the historical and physical motivation for symplectic geometry.
In the notes for this lecture we omit some of the physical calculations for brevity as they will not be relevant going further.
We attempt to retain the physical and historical motivation behind symplectic geometry.

References for this lecture are sections 1.1 and 3.1 of \cite{intro}.
\emph{(Note: From this point onward the references are specifically for the third edition of the book, I am unsure which edition the previous ones refer to.)}

\subsubsection*{Hamiltonian mechanics}
Consider a physical system, the configurations of which may be described by a point $x\in\bR^n$.
For example, the positions of three celestial objects may be described by a point in $\bR^9$.
We are interested in trajectories in this space, denoted by $t\mapsto x(t)$.
Now suppose that we have a function
\[L:\bR\times\bR^n\times\bR^n\to\bR,\quad L=L(t,x,v)\]
such that trajectories are the critical points of the "action functional":
\[I(x):=\int_{t_0}^{t_1}L(t,x(t),\dot x(t))\dee t.\]
In fact, such a function models the difference between the kinetic and potential energies of the system.
\emph{(Note: In the tradition of the physicists we will abuse notation and denote $L(t,x(t),v(t))$ by $L(t,x,v)$.)}

\begin{lem}[]
    A minimal path $x:[t_0,t_1]\to\bR^n$ satisfies the Euler-Lagrange equations:
    \[\frac{\dee}{\dee t}\frac{\partial L}{\partial v}=\frac{\partial L}{\partial x},\]
    where by $\partial L/\partial v$ we refer to $(\partial L/\partial v_1,\ldots,\partial L/\partial v_n)$, and similarly for $x$.
\end{lem}

\begin{proof}
    See Lemma 1.1.1 in \cite{intro}.
\end{proof}

This is the Lagrangian formulation of mechanics.
To transform this into the Hamiltonian formulation of mechanics we apply the Legendre transformation to our coordinates.
We replace $v_i$ by $y_i$ where
\[y_i=\frac{\partial L}{\partial v_i}(x,v).\]
This is a valid coordinate transformation as long as the Legendre condition,
\[\det\left(\frac{\partial^2L}{\partial v_i\partial v_j}\right)_{ij}\neq0,\]
                      holds.
For clarity of notation we denote $G_i(t,x,y)=v_i$, and avoid reference to $v$ so that we may take $y$ as a given.

\begin{dfn}[Hamiltonian]
    We define the Hamiltonian to be
    \[H(t,x,y)=\sum_{i=1}^n y_i\cdot G_i(t,x,y)-L(t,x,G(t,x,y)).\]
    This represents the total energy within the system.
\end{dfn}

We omit the computation but one may see that
\[\frac{\partial H}{\partial x_i}=-\dot y_i\quad\text{and}\quad\frac{\partial H}{\partial y_i}=\dot x_i.\]
These are called Hamilton's equations.
Note that while the Lagrangian formalism is concerned with vectors in space and the tangent bundle, in Hamilton's reformulation we focus on the cotangent bundle.

Write $z(t)=(x(t),y(t))$.
Then we may rewrite Hamilton's equations as
\[\dot z=-J_0\nabla H(z),\quad J_0=\begin{bmatrix}
    0&-\id_n\\ \id_n&0
\end{bmatrix}.\]

\subsubsection*{The Poisson bracket}
In what follows we assume that $H$ is independent of $t$, so $H(t,x(t),y(t))=H(x(t),y(t))$.
\todo{finish this lecture}